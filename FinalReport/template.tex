\documentclass[sigchi-a, authorversion]{acmart}
\usepackage{booktabs} % For formal tables
\usepackage{ccicons}  % For Creative Commons citation icons

% Copyright
\setcopyright{none}
%\setcopyright{acmcopyright}
%\setcopyright{acmlicensed}
%\setcopyright{rightsretained}
%\setcopyright{usgov}
%\setcopyright{usgovmixed}
%\setcopyright{cagov}
%\setcopyright{cagovmixed}


% DOI
\acmDOI{}

% ISBN
%\acmISBN{123-4567-24-567/08/06}

%Conference
\acmConference[FIS'18]{Fachpraktikum Interaktive Systeme}{July 2018}{Stuttgart, Germany}
\acmYear{2018}
\copyrightyear{2018}

\acmPrice{00.00}

%\acmBadgeL[http://ctuning.org/ae/ppopp2016.html]{ae-logo}
%\acmBadgeR[http://ctuning.org/ae/ppopp2016.html]{ae-logo}

\begin{document}
\title{Knuckle Input}

\author{Robin Schweigert}
\affiliation{%
  \institution{University of Stuttgart}
  \city{Stuttgart}
  \country{Germany} }
\email{author1@anotherco.edu}

\author{Simon Hagenmayer}
\affiliation{%
  \institution{University of Stuttgart}
  \city{Stuttgart}
  \country{Germany} }
\email{author2@author.ac.uk}

\author{Jan Leusmann}
\affiliation{%
  \institution{University of Stuttgart}
  \city{Stuttgart}
  \country{Germany} }
\email{st112158@stud.uni-stuttgart.de}


% The default list of authors is too long for headers.
\renewcommand{\shortauthors}{F. Author et al.}


%
% The code below should be generated by the tool at
% http://dl.acm.org/ccs.cfm
% Please copy and paste the code instead of the example below.
%


\begin{CCSXML}
<ccs2012>
 <concept>
<concept_id>10003120.10003121.10003122.10003334</concept_id>
<concept_desc>Human-centered computing~User studies</concept_desc>
<concept_significance>500</concept_significance>
</concept>
<concept>
<concept_id>10003120.10003138.10003141</concept_id>
<concept_desc>Human-centered computing~Ubiquitous and mobile devices</concept_desc>
<concept_significance>500</concept_significance>
</concept>
</ccs2012>
\end{CCSXML}

\ccsdesc[500]{Human-centered computing~User studies}
\ccsdesc[500]{Human-centered computing~Ubiquitous and mobile devices}


\begin{abstract}
Abstracts should be about 150 words. Required. See: \url{https://users.ece.cmu.edu/~koopman/essays/abstract.html}
\end{abstract}


\keywords{Authors' choice; of terms; separated; by semicolons; include commas, within terms only; required.}

\maketitle

\section{Introduction \& Related Work}
The structure of the final submission is only a suggestion, feel free to change if it needed. Final a example publication under: \url{http://sven-mayer.com/wp-content/uploads/2017/03/le2016placement.pdf}. How to report machine learning? See: \url{http://sven-mayer.com/wp-content/uploads/2018/01/le2018palmtouch.pdf} or \url{http://sven-mayer.com/wp-content/uploads/2017/08/mayer2017orientation.pdf}.

Motivate your project by reporting about related work and common goals.

Random example citation \cite{Le:2018:PalmTouch}.

\section{Data Collection Study}
For 
Here report: How and why is the data collection study designed as it is?
\subsection{Apparatus}
Report about your setup.
\subsection{Tasks}
Report about the tasks participants has to perform.
\subsection{Procedure}
Report about the whole procedure.
\subsection{Participants}
Report about age, gender, and similar demographics.
\section{Results}
Report about your model. No source code!

Report about the validation dataset / validation study. 
\section{Discussion}
Disuses why it is still not awesome and how this could be improved. Why this is still awesome? Think about: Nobody has done this before. 
\section{Conclusion}
Two sentences wrap up what you have done. Than report what you achieved. 


\begin{sidebar}
  \textbf{Good Utilization of the Side Bar}

  \textbf{Preparation:} Do not change the margin
  dimensions and do not flow the margin text to the
  next page.

  \textbf{Materials:} The margin box must not intrude
  or overflow into the header or the footer, or the gutter space
  between the margin paragraph and the main left column.

  \textbf{Images \& Figures:} Practically anything
  can be put in the margin if it fits. Use the
  \texttt{{\textbackslash}marginparwidth} constant to set the
  width of the figure, table, minipage, or whatever you are trying
  to fit in this skinny space.

  \caption{This is the optional caption}
  \label{bar:sidebar}
\end{sidebar}



\begin{marginfigure}
    %\includegraphics[width=\marginparwidth]{cats}
    \caption{In this image, the cats are tessellated within a square
      frame. Images should also have captions and be within the
      boundaries of the sidebar on page~\pageref{bar:sidebar}. Photo:
      \cczero~jofish on Flickr.}
    \label{fig:marginfig}
\end{marginfigure}

\begin{margintable}
    \caption{A simple narrow table in the left margin
      space.}
    \label{tab:table2}
    \begin{tabular}{r r l}
      & {\small \textbf{First}}
      & {\small \textbf{Location}} \\
      \toprule
      Child & 22.5 & Melbourne \\
      Adult & 22.0 & Bogot\'a \\
      \midrule
      Gene & 22.0 & Palo Alto \\
      John & 34.5 & Minneapolis \\
      \bottomrule
    \end{tabular}
\end{margintable}

\bibliography{bibliography}
\bibliographystyle{ACM-Reference-Format}

\end{document}
